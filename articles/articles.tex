\documentclass[a4paper,12pt,preview,tikz]{report} 
\usepackage[T2A]{fontenc}
\usepackage{tikz}
\usepackage{fancyvrb}
\usetikzlibrary{datavisualization}
\usetikzlibrary{datavisualization.formats.functions}
\usetikzlibrary{shapes,positioning,intersections,quotes}
\usepackage{changepage}
\usepackage{hyperref}
\hypersetup{
    colorlinks=true,
    linkcolor=blue,
    filecolor=magenta,      
    urlcolor=blue,
}
\usepackage{lipsum}
\usepackage{upgreek}
\usepackage[labelsep=period]{caption}
\usepackage[russian]{babel}     %russian support
\usepackage{amsmath}
\usepackage{textcomp}
\usepackage[utf8]{inputenc}
\usepackage{amssymb,amsfonts,amsmath,mathtext,cite,enumerate,float} 
\usepackage{graphicx} 
\usepackage{ragged2e}
\usepackage{indentfirst}
\usepackage{titlesec}
\usepackage{listings}
\usepackage{xcolor}
\lstset{
    language=C++,
    breaklines=true, 
    numbers=left,
    basicstyle=\ttfamily,
    language=C++,
    keywordstyle=\color{blue},
    stringstyle=\color{red},
    commentstyle=\color{green},
    morecomment=[l][\color{magenta}]{\#},
    showspaces=false,                
    showstringspaces=false,
}

\lstset{
  literate={а}{{\selectfont\char224}}1
           {б}{{\selectfont\char225}}1
           {в}{{\selectfont\char226}}1
           {г}{{\selectfont\char227}}1
           {д}{{\selectfont\char228}}1
           {е}{{\selectfont\char229}}1
           {ё}{{\"e}}1
           {ж}{{\selectfont\char230}}1
           {з}{{\selectfont\char231}}1
           {и}{{\selectfont\char232}}1
           {й}{{\selectfont\char233}}1
           {к}{{\selectfont\char234}}1
           {л}{{\selectfont\char235}}1
           {м}{{\selectfont\char236}}1
           {н}{{\selectfont\char237}}1
           {о}{{\selectfont\char238}}1
           {п}{{\selectfont\char239}}1
           {р}{{\selectfont\char240}}1
           {с}{{\selectfont\char241}}1
           {т}{{\selectfont\char242}}1
           {у}{{\selectfont\char243}}1
           {ф}{{\selectfont\char244}}1
           {х}{{\selectfont\char245}}1
           {ц}{{\selectfont\char246}}1
           {ч}{{\selectfont\char247}}1
           {ш}{{\selectfont\char248}}1
           {щ}{{\selectfont\char249}}1
           {ъ}{{\selectfont\char250}}1
           {ы}{{\selectfont\char251}}1
           {ь}{{\selectfont\char252}}1
           {э}{{\selectfont\char253}}1
           {ю}{{\selectfont\char254}}1
           {я}{{\selectfont\char255}}1
           {А}{{\selectfont\char192}}1
           {Б}{{\selectfont\char193}}1
           {В}{{\selectfont\char194}}1
           {Г}{{\selectfont\char195}}1
           {Д}{{\selectfont\char196}}1
           {Е}{{\selectfont\char197}}1
           {Ё}{{\"E}}1
           {Ж}{{\selectfont\char198}}1
           {З}{{\selectfont\char199}}1
           {И}{{\selectfont\char200}}1
           {Й}{{\selectfont\char201}}1
           {К}{{\selectfont\char202}}1
           {Л}{{\selectfont\char203}}1
           {М}{{\selectfont\char204}}1
           {Н}{{\selectfont\char205}}1
           {О}{{\selectfont\char206}}1
           {П}{{\selectfont\char207}}1
           {Р}{{\selectfont\char208}}1
           {С}{{\selectfont\char209}}1
           {Т}{{\selectfont\char210}}1
           {У}{{\selectfont\char211}}1
           {Ф}{{\selectfont\char212}}1
           {Х}{{\selectfont\char213}}1
           {Ц}{{\selectfont\char214}}1
           {Ч}{{\selectfont\char215}}1
           {Ш}{{\selectfont\char216}}1
           {Щ}{{\selectfont\char217}}1
           {Ъ}{{\selectfont\char218}}1
           {Ы}{{\selectfont\char219}}1
           {Ь}{{\selectfont\char220}}1
           {Э}{{\selectfont\char221}}1
           {Ю}{{\selectfont\char222}}1
           {Я}{{\selectfont\char223}}1
}


\usepackage{geometry} % Меняем поля страницы
\geometry{left=2cm}% левое поле
\geometry{right=1.5cm}% правое поле
\geometry{top=1cm}% верхнее поле
\geometry{bottom=2cm}% нижнее поле


\titleformat{\chapter}[hang]
{\normalfont\huge\bfseries}{\thechapter.}{20pt}{}

\begin{document}

%\begin{center}
\chapter{МЫСЛИ И ПЛАНЫ ПО ДИССЕРУ}
%\end{center}

\section{Статьи}

\textbf{ПОСМОТРЕТЬ НИЗ ВИКИ-СТРАНИЦЫ!!!!}

\subsection{DAPRA}

\textbf{Прочитать}
\begin{enumerate}
    \item \href{http://acl.mit.edu/papers/LeonardJFR08.pdf}{DARPA: A Perception-Driven Autonomous Urban Vehicle. MIT} 
    \item \href{https://www.ri.cmu.edu/pub_files/pub4/urmson_christopher_2008_1/urmson_christopher_2008_1.pdf}{Граф + Anytime D* (CMU’s Boss)}
    \item \href{https://www.romela.org/wp-content/uploads/2015/05/Odin-Team-VictorTango%E2%80%99s-Entry-in-the-DARPA-Urban-Challenge.pdf}{Граф + A* (Virginia Tech)}

\end{enumerate}

\subsection{RRT}

\textbf{Прочитаны}
\begin{enumerate}
    \item \href{https://arxiv.org/pdf/1105.1186.pdf}{Sampling-based Algorithms for Optimal Motion Planning}

        Базовое описание алгоритмов. PRM, sPRM, k-sPRM, RRT, PRM*, k-PRM*, RRG, k-RRG, RRT*, k-RRT*. 
\end{enumerate}

\textbf{Прочитать}
\begin{enumerate}
    \item \href{https://users.aalto.fi/~hamalap5/FutureGameAnimation/p113-naderi.pdf}{RT-RRT*: A Real-Time Path Planning Algorithm Based On RRT*}
    \item \href{http://idm-lab.org/bib/abstracts/papers/icra16.pdf}{RRT-Based Nonholonomic Motion Planning Using Any-Angle Path Biasing} 

\end{enumerate}

\subsection{Графовые методы}

\textbf{Прочитал}
\begin{enumerate}
    \item \href{https://en.wikipedia.org/wiki/A*_search_algorithm}{A-star} 

        Дейкстра с эвристикой

    \item \href{https://en.wikipedia.org/wiki/Any-angle_path_planning}{Any-angle}

        Можем ехать по гриду не в 8 сторон, а на любой угол.

        Есть несколько алгоритмов, которые позволяют это делать, они основаны на А*:

        \begin{enumerate}
            \item Field D*, 3D Field D*, Multi-resulutional Field D*
            \item Theta*, strict Theta*
            \item Block A*
            \item ANYA
            \item CWave
        \end{enumarate}

    \item \href{https://web.mit.edu/16.412j/www/html/papers/original_dstar_icra94.pdf}{D-star}

        Умеем во время планирования обрабатывать изменения стоимости ребер

    \item \href{https://www.ri.cmu.edu/pub_files/pub1/stentz_anthony__tony__1995_1/stentz_anthony__tony__1995_1.pdf}{Focused D*}

        Умеем немного сокращать пространство поиска в D*

    \item \href{http://idm-lab.org/bib/abstracts/papers/tro05.pdf}{D* Lite}

        Типа LPA*, умеем кэшировать стейт поиска и менять его иногда

    \item \href{https://www.ri.cmu.edu/pub_files/pub4/ferguson_david_2006_3/ferguson_david_2006_3.pdf}{Field D*}

        Умеем делать путь более прямым

    \item \href{https://aaai.org/Papers/JAIR/Vol28/JAIR-2808.pdf}{Anytime A*}

        Умеем находить субоптимальное решение, а затем его улучшать

    \item \href{https://proceedings.neurips.cc/paper/2003/file/ee8fe9093fbbb687bef15a38facc44d2-Paper.pdf}{ARA*}

        Умеем улучшать субоптимальное решение

    \item \href{https://aaai.org/Papers/ICAPS/2005/ICAPS05-027.pdf}{Anytime Dynamic A*}

        Умеем репланнить предыдущий найденный путь, использовать его на текущем тике

    \item \href{http://idm-lab.org/bib/abstracts/papers/aaai10b.pdf}{Lazy-theta*}

        Умеем в 3D планировать с хорошей погрешностью

    \item \href{https://webdocs.cs.ualberta.ca/~jonathan/publications/ai_publications/AAAI2011.pdf}{Block A*}

        Умеем в датабазе хранить стейт А*??


\end{enumerate}

\textbf{Прочитать}
\begin{enumerate}
    \item \href{https://en.wikipedia.org/wiki/Theta*}{Theta*}

        Умеем делать any-angle A* типа

    \item \href{https://sci-hub.hkvisa.net/10.1109/ICRA.2017.7989733}{CWave} \href{https://sci-hub.hkvisa.net/10.1017/S0263574719000560}{Link2}
    \item \href{https://www.researchgate.net/publication/305175423_Optimal_Any-Angle_Pathfinding_In_Practice}{ANYA}

\end{enumerate}

\subsection{Lattice Planner}

\textbf{Прочитать}
\begin{enumerate}
    \item \href{https://rpal.cs.cornell.edu/docs/PivEtal_JFR_2009.pdf}{Хорошая подробная статья про мобильных роботов на основе lattice-а без полос.} 
    \item \href{https://en.wikipedia.org/wiki/Any-angle_path_plannin://arxiv.org/pdf/1809.02399.pdf}{Интересный вариант совмещение RRT и lattice идей.}
    \item \href{http://persoal.citius.usc.es/manuel.mucientes/pubs/Gonzalez-Sieira18_ras.pdf}{Интересная идея с сеткой разной градации}

\end{enumerate}


\end{document}


